%% LyX 2.3.4.2 created this file.  For more info, see http://www.lyx.org/.
%% Do not edit unless you really know what you are doing.
\documentclass[12pt,english]{article}
\usepackage[T1]{fontenc}
\usepackage{geometry}
\geometry{verbose,tmargin=2.5cm,bmargin=2.5cm,lmargin=2.5cm,rmargin=2.5cm}
\setcounter{secnumdepth}{4}
\setcounter{tocdepth}{4}
\usepackage{babel}
\usepackage{refstyle}
\usepackage{float}
\usepackage{amstext}
\usepackage{graphicx}
\usepackage[unicode=true,pdfusetitle,
 bookmarks=true,bookmarksnumbered=false,bookmarksopen=false,
 breaklinks=false,pdfborder={0 0 1},backref=false,colorlinks=false]
 {hyperref}

\makeatletter

%%%%%%%%%%%%%%%%%%%%%%%%%%%%%% LyX specific LaTeX commands.

\AtBeginDocument{\providecommand\figref[1]{\ref{fig:#1}}}
\AtBeginDocument{\providecommand\subsecref[1]{\ref{subsec:#1}}}
\AtBeginDocument{\providecommand\parref[1]{\ref{par:#1}}}
\AtBeginDocument{\providecommand\tabref[1]{\ref{tab:#1}}}
\AtBeginDocument{\providecommand\secref[1]{\ref{sec:#1}}}
\floatstyle{ruled}
\newfloat{algorithm}{tbp}{loa}
\providecommand{\algorithmname}{Algorithm}
\floatname{algorithm}{\protect\algorithmname}
\RS@ifundefined{subsecref}
  {\newref{subsec}{name = \RSsectxt}}
  {}
\RS@ifundefined{thmref}
  {\def\RSthmtxt{theorem~}\newref{thm}{name = \RSthmtxt}}
  {}
\RS@ifundefined{lemref}
  {\def\RSlemtxt{lemma~}\newref{lem}{name = \RSlemtxt}}
  {}


%%%%%%%%%%%%%%%%%%%%%%%%%%%%%% Textclass specific LaTeX commands.
\newenvironment{lyxcode}
	{\par\begin{list}{}{
		\setlength{\rightmargin}{\leftmargin}
		\setlength{\listparindent}{0pt}% needed for AMS classes
		\raggedright
		\setlength{\itemsep}{0pt}
		\setlength{\parsep}{0pt}
		\normalfont\ttfamily}%
	 \item[]}
	{\end{list}}

%%%%%%%%%%%%%%%%%%%%%%%%%%%%%% User specified LaTeX commands.
\newcommand{\indep}{\perp \!\!\! \perp}

\makeatother

\begin{document}
\title{Some Project Title}
\author{Name Nameson, Name Namesberg}
\maketitle

\section{Abstract}

Up to 250 words, similar to abstracts in articles.

\section{Introduction}
Two to three written pages, figures are strongly encouraged but are not included in the page count. 
Needs to include a description of the problem and the data and of the research question in hand.

\subsection{Background}

Some general background to the problem.

\subsubsection{A Possible Dive Into a Specific Relevant Topic}



\subsubsection{Another Possible Dive}



\subsection{Related Work}

Covering relevant past research and works, should include citations in a clear and consistent format.

\subsection{Data}
 

\section{Results}

A description of the research results.
Try to make it clear how the results you show relate to the research questions, add figures.
We recommend using different subsections for different results,

\subsection{An Informative Subsection Title Regarding Some Experiments and Results}

Accompanied by relevant plots, figures and tables.

\subsection{Another Informative Subsection Title Regarding Some Different Experiments and Results}
Also containing relevant visuals.

\section{Discussion}
About 2 written pages, discussing the results and their implications.

\subsection{A Brief Summary of the Results in Retrospective }

\subsection{Conclusions Based on the Results}


\subsection{Limitations}

A critical discussion on the work's limitations, simplifying assumptions, possible biases etc, and their possible effects on the results and the conclusions.


\subsection{Future Work}
A discussion on relevant future work, maybe taking into account the current limitations and possible solutions to them. 



\section{Resources and Methods}

\subsection{Preprocessing}



\subsection{Statistical Methods and Considerations}


\subsection{Models and ML Methods Used}
If they were not thoroughly discussed earlier. 
Maybe relevant model training details.

\subsection{Resources}


\subsection{Work Pipeline}
A description and a flowchart of the work pipeline 


\subsection*{Division of work}
If you split the responsibilities. 
\begin{itemize}
\item Name Nameson: Some work done only by them.
\item Name Namesberg: Some work done only by them.
\end{itemize}
\begin{thebibliography}{1}
\bibitem{key-8}Settles, B. (2009). Active Learning Literature Survey
(Computer Sciences Technical Report No. 1648). University of Wisconsin--Madison. 

\bibitem{key-9}Donggeun Yoo and In So Kweon. Learning loss for active
learning. In CVPR, pages 93--102, 2019.

\bibitem{key-10}Nicolas Carion, Francisco Massa, Gabriel Synnaeve,
Nicolas Usunier, Alexander Kirillov, and Sergey Zagoruyko. End-to-end
object detection with transformers. In ECCV, 2020

\bibitem{key-11}Mark Everingham, Luc Van Gool, Christopher KI Williams,
John Winn, and Andrew Zisserman. The pascal visual object classes
(voc) challenge. International journal of computer vision, 88(2):303--338,
2010

\bibitem{key-3}Andreas Geiger, Philip Lenz, and Raquel Urtasun. Are
we ready for autonomous driving? the kitti vision benchmark suite.
In Conference on Computer Vision and Pattern Recognition (CVPR), 2012.

\bibitem{key-12}Xin Wang, Yudong Chen, and Wenwu Zhu. 2021. A survey
on curriculum learning. IEEE Transactions on Pattern Analysis and
Machine Intelligence (TPAMI)

\bibitem{key-4}yukkyo (2021). voc2coco. GitHub. Note: https://github.com/yukkyo/voc2coco. 

\bibitem{key-5}Moore, B., \& Corso, J. (2020). FiftyOne. GitHub.
Note: https://github.com/voxel51/fiftyone. 

\bibitem{key-6}McInnes, L., Healy, J., \& Melville, J. (2018). UMAP:
Uniform Manifold Approximation and Projection for Dimension Reduction.
ArXiv e-prints.
\end{thebibliography}

\end{document}
